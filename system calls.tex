\section{Lab: System calls}

\subsection{实验目的}

本实验旨在进一步熟悉系统调用,重点掌握如何添加系统调用,理解系统调用的工作原理,内核态和用户态的联系。

本实验实现了两个系统调用:
\begin{itemize}
    \item System call tracing:实现对系统调用的跟踪
    \item Sysinfo:收集有关正在运行的系统的信息
\end{itemize}

\subsection{实验步骤}

\subsubsection{实现trace}

\begin{enumerate}
    \item 首先,根据提示,在Makefile的UPROGS中添加\$U/\_trace\textbackslash。
    \item 在user.h文件的system calls部分添加trace函数。
          \newpage
          \begin{lstlisting}[language=c, title=对user.h的改动]
    // system calls
    int fork(void);
    ...
    // 添加 trace函数
    int trace(int);
    \end{lstlisting}
    \item 在usys.pl末尾添加entry("trace");以便在汇编语言中添加这个函数。
    \item 进入kernel文件夹,在syscall.h中添加SYS\_trace系统调用号的定义,值往下类推,即为22。
          \begin{lstlisting}[language=c, title=对syscall.h的更改]
    // System call numbers
    #define SYS_fork 1
    ...
    // 为trace添加系统调用号
    #define SYS_trace 22
          \end{lstlisting}
    \item 更改进程结构体。在proc.h中的进程结构体最后添加一个掩码trace\_mask,用于记录哪些进程要被追踪。
          \begin{lstlisting}[language=c, title=对进程结构体的更改]
    struct proc
    {
        struct spinlock lock;
        ...
        // 添加掩码
        int trace_mask;
    };
    \end{lstlisting}
    \item 在sysproc.c文件中,添加系统调用函数sys\_trace,在调用时设置掩码。
          \begin{lstlisting}[language=c, title=实现sys\_trace系统调用]
    uint64 sys_trace(void)
    {
        // 获取参数
        int mask;
        if (argint(0, &mask) < 0)
            return -1;
        // 获取当前进程
        struct proc *p = myproc();
        // 写入掩码
        p->trace_mask = mask;
        return 0;
    }
    \end{lstlisting}
    \item 在syscall.c中修改syscall函数,使其根据掩码判断是否输出当前调用信息。
          \begin{lstlisting}[language=c, title=对syscall函数的修改]
    void syscall(void)
    {
        int num;
        struct proc *p = myproc();

        num = p->trapframe->a7;
        if (num > 0 && num < NELEM(syscalls) && syscalls[num])
        {
            p->trapframe->a0 = syscalls[num]();

            // 获得掩码
            int trace_mask = p->trace_mask;

            // 如果掩码对应,输出当前系统调用信息
            if ((trace_mask >> num) & 1)
            {
                printf("%d: syscall %s -> %d\n", 
                        p->pid, 
                        syscall_names[num - 1], 
                        p->trapframe->a0);
            }
        }
        else
        {
            ...
        }
    }
    \end{lstlisting}
    \item 修改proc.c中的freeproc函数和fork函数,确保掩码在进程释放时重置掩码、在子进程创建时传递掩码。
          \newpage
          \begin{lstlisting}[language=c, title=对proc.c的修改]
    static void freeproc(struct proc *p)
    {
        if (p->trapframe)
        kfree((void *)p->trapframe);
        ...

        // 添加了这一行
        p->trace_mask = 0;
    }

    int fork(void)
    {
        ...

        // 把子进程的掩码也设置为与父进程相同
        np->trace_mask = p->trace_mask;

        ...
    }
    \end{lstlisting}
\end{enumerate}